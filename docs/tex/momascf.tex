% MOMASCF Documentation
\documentclass{article}
\usepackage{amsmath}
\usepackage{amssymb}
\usepackage{graphicx}

\title{MOMASCF: A Categorical Framework for Monetary and Financial Systems}
\author{Viktor Winschel}
\date{\today}

\begin{document}

\maketitle

\section{Introduction}

This document describes the categorical structure of the MOMASCF system.

\section{Basic Structure}

The basic categorical structure is given by:

\[
\begin{align*}
\cat{C} &= (\mathrm{Obj}(\cat{C}), \mathrm{Mor}(\cat{C}), \circ, \mathrm{id}) \\
\mathrm{Obj}(\cat{C}) &= \{\text{Accounts}\} \cup \{\text{Agents}\} \\
\mathrm{Mor}(\cat{C}) &= \{f: A \to B \mid A, B \in \mathrm{Obj}(\cat{C})\}
\end{align*}
\]

\section{Double-Entry Structure}

The double-entry principle is represented by the following pullback:

\[
\begin{array}{ccc}
\text{Transaction} & \xrightarrow{\text{credit}} & \text{CreditAccount} \\
\downarrow_{\text{debit}} & & \downarrow_{\text{balance}} \\
\text{DebitAccount} & \xrightarrow{\text{balance}} & \text{Money}
\end{array}
\]

\section{Balance Sheet Functor}

The balance sheet structure is represented by a functor:

\[
\begin{array}{ccc}
\cat{C}(t) & \xrightarrow{F_t} & \cat{C}(t+1) \\
\downarrow_{\beta_t} & & \downarrow_{\beta_{t+1}} \\
\Ab & \xrightarrow{\id} & \Ab
\end{array}
\]

\end{document} 