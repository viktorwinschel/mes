\documentclass{article}
\usepackage{amsmath}
\usepackage{amssymb}
\usepackage{graphicx}
\usepackage{listings}
\usepackage{hyperref}
\usepackage{xcolor}

\title{Monetary Macro-Accounting Model for Supply Chain Finance (MoMaSCF)}
\author{Documentation}
\date{\today}

\begin{document}

\maketitle

\section{Overview}

MoMaSCF is a monetary macro-accounting model that simulates financial flows and economic interactions in a supply chain system. The model represents a closed economy with four main agents (Labor, Resource Owners, Company, and Capitalists) interacting through a banking system.

\subsection{Historical Context}
\begin{itemize}
    \item Historical bill market in Germany 1990: 50 Billion DM per annum
    \item Current trade finance market: $\sim 10^{12}$, with 2/3 as open account (deferred payment)
\end{itemize}

\section{Model Structure}

\subsection{Agents and Their Roles}

\subsubsection{Labor (Households)}
\begin{itemize}
    \item Provides labor to companies
    \item Receives wages
    \item Consumes goods
    \item Maintains bank accounts and goods inventory
\end{itemize}

\subsubsection{Resource Owners}
\begin{itemize}
    \item Supply raw materials
    \item Receive resource payments
    \item Consume goods
    \item Maintain resource stocks and bank accounts
\end{itemize}

\subsubsection{Company (Producer)}
\begin{itemize}
    \item Produces goods using labor and resources
    \item Takes loans for investment
    \item Pays wages, resource costs, and dividends
    \item Maintains multiple account types (bank, loan, inventory)
\end{itemize}

\subsubsection{Capitalists}
\begin{itemize}
    \item Receive dividends
    \item Consume goods
    \item Maintain bank accounts and goods inventory
\end{itemize}

\subsubsection{Bank}
\begin{itemize}
    \item Provides loans to companies
    \item Maintains accounts for all agents
    \item Facilitates payments and settlements
\end{itemize}

\section{Mathematical Framework}

\subsection{Key Parameters}
\begin{align*}
    \text{InvestmentLen} &= 10 && \text{(Investment cycle length)} \\
    \text{LabResourceRatio} &= 0.2 && \text{(Labor to resources ratio)} \\
    \text{ConsumRatio}_i &\in \{0.8, 0.95, 0.6\} && \text{(Consumption ratios)} \\
    \text{MarkUp} &= 0.5 && \text{(Price markup)} \\
    \text{DivRateDiff} &= 0.15 && \text{(Dividend ratio from surplus)} \\
    \text{DivRateBank} &= 0.4 && \text{(Dividend ratio from balance)}
\end{align*}

\subsection{Production Function}
The production function follows a Cobb-Douglas form:
\begin{equation}
    Q = 1 + \text{ScaleProd} \cdot L^{\alpha} \cdot R^{1-\alpha}
\end{equation}
where:
\begin{itemize}
    \item $Q$: Production output
    \item $L$: Labor input
    \item $R$: Resource input
    \item $\alpha = \text{LabResSubstProd} = 0.75$
    \item $\text{ScaleProd} = 0.42$
\end{itemize}

\subsection{Investment Function}
Investment follows a sigmoid response to demand surplus:
\begin{equation}
    I = \text{sigA} + \frac{\text{sigB}}{1 + e^{-D_s/\text{sigC}}}
\end{equation}
where:
\begin{itemize}
    \item $D_s$: Demand surplus
    \item $\text{sigA} = 20.0$ (Base level)
    \item $\text{sigB} = 480.0$ (Scale)
    \item $\text{sigC} = 200.0$ (Sensitivity)
\end{itemize}

\subsection{Price Formation}
Price is determined through categorical morphisms:
\begin{equation}
    P = P_0 \cdot \mathbb{1}_{t=0} + \frac{D_p}{Q} + W_p \cdot D_s \cdot \mathbb{1}_{D_s > 0}
\end{equation}
where:
\begin{itemize}
    \item $P_0$: Initial price
    \item $D_p$: Planned demand
    \item $Q$: Production
    \item $D_s$: Demand surplus
    \item $W_p$: Windfall profit ratio
\end{itemize}

\subsection{Dividend Decision}
Dividends are calculated using two components:
\begin{equation}
    D = r_d \cdot \max(L_d, 0) + r_b \cdot \max(B, 0)
\end{equation}
where:
\begin{itemize}
    \item $L_d$: Loan difference
    \item $B$: Bank balance
    \item $r_d$: Dividend rate from difference
    \item $r_b$: Dividend rate from balance
\end{itemize}

\section{Categorical Implementation}

\subsection{Morphisms}
The model implements categorical morphisms for decision-making:

\begin{lstlisting}[language=Julia]
struct ConditionalMorphism
    condition::Function
    true_morphism::Function
    false_morphism::Function
end

struct PriceFormationMorphisms
    initial_price::ConditionalMorphism
    demand_price::Function
    surplus_price::ConditionalMorphism
end

struct DividendDecisionMorphisms
    surplus_dividend::ConditionalMorphism
    bank_dividend::ConditionalMorphism
end
\end{lstlisting}

\section{System Dynamics}

\subsection{Flow Equations}
Key flow relationships:
\begin{align*}
    \text{FLOW\_CONSUMPTION}_i &= \text{BANK}_i \cdot \text{ConsumRatio}_i \\
    \text{FLOW\_INVESTMENT\_RESOURCE} &= I \cdot (1 - \text{LabResourceRatio}) \\
    \text{FLOW\_INVESTMENT\_LABOR} &= I \cdot \text{LabResourceRatio}
\end{align*}

\subsection{Clearing System}
Two-phase clearing with efficiency rates:
\begin{align*}
    \text{MORNING\_NETTED} &= 0.85 \cdot \text{MORNING\_FLOWS} \\
    \text{EVENING\_NETTED} &= 0.90 \cdot \text{EVENING\_FLOWS}
\end{align*}

\section{Results}

\subsection{Key Metrics}
Average values from simulation:
\begin{align*}
    \text{Wage Payment} &\approx 877.77 \\
    \text{Resource Payment} &\approx 438.88 \\
    \text{Dividend Payment} &\approx 2.63 \times 10^6 \\
    \text{Production} &\approx 21.62 \\
    \text{Price} &\approx 2.39 \times 10^6
\end{align*}

\subsection{System Stability}
The model maintains:
\begin{itemize}
    \item Net positions near equilibrium
    \item Consistent clearing efficiencies
    \item Balanced flow relationships
\end{itemize}

\section{Conclusion}

MoMaSCF successfully implements a categorical approach to monetary macro-accounting, demonstrating:
\begin{itemize}
    \item Robust financial flow modeling
    \item Stable clearing mechanisms
    \item Complex agent interactions
    \item Emergent economic behavior
\end{itemize}

\end{document} 