\documentclass{article}
\usepackage{amsmath, amssymb}

\title{Memory Evolutive Systems - Mathematical Formulas}
\author{}
\date{}

\begin{document}

\maketitle

\section{Colimits in Categories}
A colimit captures the idea of binding objects and morphisms into a new higher-level object:
\begin{equation}
    \forall X, \exists! \psi: C \to X \text{ such that } \psi \circ \varphi_D = \text{unique}
\end{equation}
where $\varphi_D$ represents a universal morphism integrating a pattern.

\section{Functorial Evolution of Systems}
A functor $F: \mathcal{C}(t) \to \mathcal{C}(t+1)$ models the transformation of system configurations over time:
\begin{equation}
    F(A) = A' \quad \text{and} \quad F(f: A \to B) = f': A' \to B'
\end{equation}
This ensures structural preservation during evolution.

\section{Composition and Associativity}
Composition in categories follows an associative rule:
\begin{equation}
    (f \circ g) \circ h = f \circ (g \circ h)
\end{equation}
ensuring a unique way to interpret composition sequences.

\section{Universal Property of Limits}
Limits generalize constructions like products, ensuring a unique mapping:
\begin{equation}
    \forall X, \exists! \psi: X \to L \text{ such that } \varphi_X = \psi \circ \varphi_L
\end{equation}
where $L$ is the limit object.

\section{Complexification Process}
Successive complexifications form a hierarchy:
\begin{equation}
    C_0 \to C_1 \to C_2 \to \dots \to C_n
\end{equation}
Each step integrates patterns into a higher-order structure.

\section{Multiplicity Principle}
Emergence requires degeneracy in colimits:
\begin{equation}
    \exists P, Q \text{ such that } colim(P) = colim(Q) 
\end{equation}
indicating that different substructures can form equivalent emergent objects.

\section{Memory Dynamics}
The evolution of memory structures can be functorially expressed as:
\begin{equation}
    M_{t+1} = F(M_t, P_t)
\end{equation}
where $M_t$ is the memory at time $t$, and $P_t$ represents new procedural inputs.

\end{document}
