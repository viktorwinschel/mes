\documentclass{article}
\usepackage{amsmath}
\usepackage{amssymb}
\usepackage{graphicx}

\title{Memory Evolutive Systems: A Categorical Framework for Complex Systems}
\author{Viktor Winschel}
\date{\today}

\begin{document}

\maketitle

\section{Introduction}

This paper presents a categorical framework for complex systems based on Memory Evolutive Systems.

\section{Basic Structure}

The hierarchical structure of a Memory Evolutive System is represented by:

\[
\begin{array}{ccccc}
\cat{C}_0 & \xrightarrow{F_1} & \cat{C}_1 & \xrightarrow{F_2} & \cat{C}_2
\end{array}
\]

where each level represents increasing complexity and abstraction.

\section{Colimit Construction}

The binding process is represented by colimits:

\[
\begin{array}{ccc}
\text{Transaction} & \xrightarrow{\text{credit}} & \text{CreditAccount} \\
\downarrow_{\text{debit}} & & \downarrow_{\text{balance}} \\
\text{DebitAccount} & \xrightarrow{\text{balance}} & \text{Money}
\end{array}
\]

\section{Complex Patterns}

Complex patterns emerge through natural transformations:

\[
\begin{array}{ccc}
\cat{C}(t) & \xrightarrow{F_t} & \cat{C}(t+1) \\
\downarrow_{\alpha_t} & & \downarrow_{\alpha_{t+1}} \\
\cat{D}(t) & \xrightarrow{G_t} & \cat{D}(t+1)
\end{array}
\]

\section{Balance Sheet Functor}

The balance sheet functor is defined as:

\[
\begin{array}{l}
B(\text{Account}) &= \mathbb{R} \text{ (account balance)} \\
B(f: A \to B) &= (+f_{\text{amount}}) \text{ (transaction amount)}
\end{array}
\]

\end{document}
