\documentclass{article}
\usepackage{tikz-cd}
\usepackage{amsmath}
\usepackage{amssymb}
\usepackage{hyperref}
\usepackage{cleveref}


\title{Commutative Diagrams for Algebras}
\author{Dr.\,Viktor Winschel\footnote{viktor.winschel@gmail.com, +49 179 7621055}}
\date{}

\begin{document}

\maketitle

An \emph{algebra} \((U, a)\) over a functor \(T \colon \mathcal{C} \to \mathcal{C}\) consists of a carrier object \(U\) and a structure morphism \(a \colon T(U) \to U\).

A morphism of \(T\)-algebras \(f \colon (U, a) \to (V, b)\) is a morphism \(f \colon U \to V\) in \(\mathcal{C}\) such that the following square commutes:

\begin{equation}
\label{eq:naturality}
\begin{tikzcd}[row sep=large, column sep=large]
T(U) \arrow[r,"T(f)"] \arrow[d,"a"'] & T(V) \arrow[d,"b"] \\
U \arrow[r,"f"'] & V
\end{tikzcd}
\end{equation}

That is, \(f \circ a = b \circ T(f)\), meaning \(f\) respects the algebra structures. Such morphisms are called \emph{algebra homomorphisms} or simply \emph{natural}.

These morphisms compose, giving rise to sequences in the category of \(T\)-algebras:

\begin{equation}
\label{eq:algebra-morphism-sequence}
\left( T(U) \xrightarrow{a} U \right) \xrightarrow{f}
\left( T(V) \xrightarrow{b} V \right) \xrightarrow{g}
\left( T(W) \xrightarrow{c} W \right)
\end{equation}

The category formed by \(T\)-algebras and their morphisms is called the \emph{Eilenberg–Moore category} \(\mathcal{C}^T\).

The following diagram shows that functorial composition is compatible with algebra structure:

\begin{equation}
\label{eq:composition}
\begin{tikzcd}[row sep=large, column sep=large]
T(U) \arrow[r,"T(f)"] \arrow[d,"a"'] \arrow[rr,bend left=40,"T(g \circ f)"] &
T(V) \arrow[r,"T(g)"] \arrow[d,"b"'] &
T(W) \arrow[d,"c"] \\
U \arrow[r,"f"'] \arrow[rr,bend right=40,"g \circ f"'] &
V \arrow[r,"g"'] &
W
\end{tikzcd}
\end{equation}

This diagram ensures that composition in \(\mathcal{C}^T\) is well-defined: morphisms of algebras preserve structure not just individually, but also under composition, reflecting the functoriality of \(T\) and the coherence of the algebra morphisms.

\end{document}
